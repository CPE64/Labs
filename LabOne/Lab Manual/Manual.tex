%| Weekly report template for CSUS Senior Design
%|
%| language: LaTeX
%| Author: Ben Smith
%| 
%| This source has been tagged with the "<CHANGE" tag in areas
%| that require updating when making a new docuent
%|
%| This source will generate a PDF file complete with thumbnails navigation menu and metadata.
%| Much of the tex awesomeness comes from http://www.michaelshell.org/ praise be to him for creating the guide

\documentclass[12pt,journal]{IEEEtran}

%| Override compsoc class' Palatino font for body text, restores to Times New Roman
\renewcommand{\rmdefault}{ptm}\selectfont

%| IEEE Citation package
 \usepackage{cite}

\usepackage[cmex10]{amsmath}        %| American Mathematical Society package for fancy maths  
\interdisplaylinepenalty=2500       %| Restores IEEE line spacing after amsmath

%| Better tables than LaTeX 2e
\usepackage{array}
\usepackage{graphicx}
\usepackage{float}
\usepackage{url}                    %| Improved URL handling
\usepackage{minted}
\usepackage{etoolbox}
\AtBeginEnvironment{minted}{\singlespacing%
    \fontsize{9}{p}\selectfont}

%| Enables PDF metadata, thumbnails, and navigation
\newcommand\MYhyperrefoptions{
  bookmarks=true,
  bookmarksnumbered=true,
  pdfpagemode={UseOutlines},
  plainpages=false,
  pdfpagelabels=true,
  colorlinks=true,
  linkcolor={black},
  citecolor={black},
  urlcolor={black},
  pdftitle={CPE/EEE 64: Lab One},
  pdfsubject={Engineering},                        
  pdfauthor={California State University Sacramento},
  pdfkeywords={}}                       

%| Calls hyperref package with the options specified above
\usepackage[\MYhyperrefoptions,pdftex]{hyperref}

\begin{document}
  %| =================================================================================================
  %| Title Block
  %| =================================================================================================
  \title{CPE/EEE 64: Lab One}
  \author{Introduction to Verilog}
  \date{\today}
  \maketitle

  %| =================================================================================================
  %| Weekly time report
  %| =================================================================================================
  \section{\bfseries Introduction}
    Verilog is a powerful way to describe circuits. Logic diagrams of that are being used in lecture 
    can become cumbersome as designs become large. Text based design can be less prone to error because 
    it is easier to track differences in large designs. Verilog is commonly used for ASIC design as well 
    as FPGAs. We will be using the Terasic DE0-Nano development board with an Altera Cyclone IV FPGA on 
    board. Altera provides a comprehensive solution for programming and debugging their FPGAs called Quartus. 
    These labs will explore Quartus' and use it to program the FPGA on the Nano. In this lab we will:
    \begin{itemize}
      \item Instantiate a System Verilog module
      \item Use a System Verilog Testbench
      \item Bink some LED's
    \end{itemize}
  
  \subsection{Verilog Modularity}
    One of the most important features of Verilog is it's ability to reuse a design. This section will 
    involve building logic gate modules which will be reused in later labs to build more complex
    structures. Reusing these modules is very similar to how you would reuse code in the workplace to be
    more productive. The lab documentation comes with Verilog implementations of the four logic gates in 
    {\bfseries Source.zip} The demo for the lab will be implementing these modules with the DE0-Nano development
    board and testing the design on a breadboard. You could think of this as the source libraries that would
    be available at the company that you might work for. 

  \subsection{Test Bench}
    Verilog roughly breaks into two halves synthesizable and non-synthesizable. FPGAs synthesis can
    take a very long time, using a simulator to verify individual modules can be much faster than
    resynthesizing the entire design. The Testbench also offers a unique ability to check expected
    outputs generate test stimulus. We will use a test bench to check the provided verilog modules
    are providing the desired operation in part C of the procedure.

  \subsection{Altera tutorials}
    Altera offers a complete traning course for the interested. You have to have a free account
    on Altera.com {\bfseries protip: do it.}

  \section{\bfseries Lab Procedure}
    \subsection{Install Quartus}
      Download most recent version of Quartus and Cyclone device drivers from Altera. Screencast 1
      is a walk through for windows.
    \subsection{Expand Source.Zip}
      The logic gate modules and test bench are contained within this archive, first one's free.
    \subsection{Run testbench}
      Open Modelsim and compile the test bench. Start a new simulation and add the waveforms as 
      shown in screencast 2.
    \subsection{Program and test the DEO-0 nano}
      \subsubsection{Create top module and assign pins} refer to screencast 3 for a walk through. Verilog wires
      will need to be assigned to output pins on the FPGA package. The DE0-Nano instruction manual has tables
      that you can look up the values you need for the assignments.
      \subsubsection{Use Quartus to program the Nano} refer to screencast 4 for a walkthrough. After synthesis
      Quartus will generate a .SOF file that can be used to program the FPGA using Quartus' programmer.
      \subsubsection{Test behavior against expected truth table} prepare a table to expected outputs
      from the known properties of the logic gate.

  \section{\bfseries  Lab Report}
    The lab report must be typed and submitted as a .PDF. Look to IEEE's guidelines for publishing for directions
    on format. Include

\end{document}