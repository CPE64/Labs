%| This is a header file for Latex documents  
%| It contains a number of common packages, settings, and custom macros that I frequently use.
\documentclass[9pt,journal]{IEEEtran}

\usepackage[cmex10]{amsmath}        %| American Mathematical Society package for fancy maths  b
\interdisplaylinepenalty=2500              %| Restores IEEE line spacing after amsmath

%| IEEE Citation package
\usepackage{cite}
\usepackage[section]{placeins}
\usepackage{array}
\usepackage{dblfloatfix}
\usepackage{color}
\usepackage{graphicx}
\usepackage{float}
\usepackage{url}                         %| Improved URL handling
\usepackage{etoolbox}
\usepackage[font=footnotesize]{subcaption}
\usepackage{listings}
\usepackage{fixltx2e}               %| Better tables than for LaTeX 2e
\usepackage{minted}

%| Highlighting for source code listings
\definecolor{mygreen}{rgb}{0,0.6,0}
\definecolor{ltgray}{rgb}{0.93,0.93,0.93}
\definecolor{dkgray}{rgb}{0.5,0.5,0.5}
\definecolor{mymauve}{rgb}{0.58,0,0.82}
\lstset{
  backgroundcolor=\color{ltgray},  % choose the background color; you must add \usepackage{color} or \usepackage{xcolor}
  basicstyle=\scriptsize\ttfamily, % the size of the fonts that are used for the code
  breakatwhitespace=true,         % sets if automatic breaks should only happen at whitespace
  breaklines=true,                 % sets automatic line breaking
  captionpos=b,                    % sets the caption-position to bottom
  commentstyle=\color{mygreen},    % comment style
  deletekeywords={...},            % if you want to delete keywords from the given language
  escapeinside={\%*}{*)},          % if you want to add LaTeX within your code
  extendedchars=true,              % lets you use non-ASCII characters; for 8-bits encodings only, does not work with UTF-8
  frame=single,                    % adds a frame around the code
  keepspaces=true,                 % keeps spaces in text, useful for keeping indentation of code (possibly needs columns=flexible)
  keywordstyle=\color{blue},       % keyword style
  language=SystemVerilog,          % the language of the code(I modified the .sty for systemverilog, found the code on google)
  morekeywords={*,...},            % if you want to add more keywords to the set
  numbers=left,                    % where to put the line-numbers; possible values are (none, left, right)
  numbersep=4pt,                   % how far the line-numbers are from the code
  numberstyle=\tiny\color{dkgray}, % the style that is used for the line-numbers
  rulecolor=\color{black},         % if not set, the frame-color may be changed on line-breaks within not-black text (e.g. comments (green here))
  showspaces=false,                % show spaces everywhere adding particular underscores; it overrides 'showstringspaces'
  showstringspaces=false,          % underline spaces within strings only
  showtabs=false,                  % show tabs within strings adding particular underscores
  stepnumber=2,                    % the step between two line-numbers. If it's 1, each line will be numbered
  stringstyle=\color{mymauve},     % string literal style
  tabsize=2,                       % sets default tabsize to 2 spaces
  title=\lstname                   % show the filename of files included with \lstinputlisting; also try caption instead of title
}

\lstset{keywordstyle=\color{purple}}
\lstset{keywordstyle={[2]\color{purple}} }
\lstset{keywordstyle={[3]\color{magenta}} }
\lstset{keywordstyle={[4]\color{teal} }}
\lstset{keywordstyle={[5]\color{violet!40}} }

% Alter some LaTeX defaults for better treatment of figures:
  % See p.105 of ''TeX Unbound'' for suggested values.
  % See pp. 199-200 of Lamport's ''LaTeX'' book for details.
  %   General parameters, for ALL pages:
  \renewcommand{\topfraction}{0.9}  % max fraction of floats at top
  \renewcommand{\bottomfraction}{0.8} % max fraction of floats at bottom
  %   Parameters for TEXT pages (not float pages):
  \setcounter{topnumber}{2}
  \setcounter{bottomnumber}{2}
  \setcounter{totalnumber}{4}     % 2 may work better
  \setcounter{dbltopnumber}{2}    % for 2-column pages
  \renewcommand{\dbltopfraction}{0.9} % fit big float above 2-col. text
  \renewcommand{\textfraction}{0.07}  % allow minimal text w. figs
  %   Parameters for FLOAT pages (not text pages):
  \renewcommand{\floatpagefraction}{0.7}  % require fuller float pages
  % N.B.: floatpagefraction MUST be less than topfraction !!
  \renewcommand{\dblfloatpagefraction}{0.7} % require fuller float pages

%| Enables PDF metadata, thumbnails, and navigation
\newcommand\MYhyperrefoptions{
  bookmarks=true,
  bookmarksnumbered=true,
  pdfpagemode={UseOutlines},
  plainpages=false,
  pdfpagelabels=true,
  colorlinks=true,
  linkcolor={black},
  citecolor={black},
  urlcolor={blue},
  pdftitle={CPE/EEE 64 Lab},
  pdfsubject={Engineering},                        
  pdfauthor={Ben Smith},
  pdfkeywords={Logic Design, FPGA, Verilog}}                       

%| Calls hyperref package with the options specified above
\usepackage[\MYhyperrefoptions,pdftex]{hyperref}

%| Font settings
\renewcommand{\sfdefault}{phv}
\renewcommand{\rmdefault}{ptm}
\renewcommand{\ttdefault}{pcr}

%| Restores IEEE table formatting after usage of subcaption package
\captionsetup[table]{format=plain,labelformat=simple,justification=centering, labelsep=newline, singlelinecheck=false, textfont={sc}}

%| Required Lab Demo custom function
%| \demo{Name}{Physical deliverable}{Documentation deliverable}{Process}
%| =================================================================================================
%| for boxed text and stuch
\usepackage{fancybox}
\newenvironment{fminipage}%
{\begin{Sbox}\begin{minipage}}%
{\end{minipage}\end{Sbox}\Ovalbox{\TheSbox}}

%| Actual bawx
\newcommand{\demo}[4] {
\vspace{15px}
\begin{centering}
  \begin{fminipage}{.47\textwidth}
    \vspace{3px}
    \centering{\bfseries \large Laboratory Demo: #1}\\*
    \vspace{10px}
    \begin{tabular}{p{1.4cm}  p{6.3cm}}
      %|==Requirements for lab demo==
      \raggedright Specification:                  &#2\\
      \\
      \raggedright  Deliverable:                   &#3\\
      \\
      \raggedright Process :                       &#4\\
    \end{tabular}
  \end{fminipage}
\end{centering}
}

%| Single figure
%| \small{Location}{Caption}{Label}
%| =================================================================================================
\newcommand{\smallfig}[3] {
  \begin{figure}[H]
    \includegraphics[width=.48\textwidth]{#1}
    \caption{#2}
    \label{#3}
  \end{figure}
}